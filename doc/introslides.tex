\documentclass[xcolor=dvipsnames]{beamer}

\usetheme{Madrid}
\setbeamertemplate{navigation symbols}{}

\usepackage{mpctools}

\title{MPC with Casadi/Python}
\date{June 10, 2015}
\author{Michael Risbeck}

\begin{document}

\frame{\titlepage}

\begin{frame}{What's in \texttt{mpc-tools-casadi}?}
    A Python package: \texttt{mpctools}\footnote{See \smallurl{install.pdf} for detailed installation instructions.}
    \begin{itemize}
        \item You should put the \texttt{mpctools} folder somewhere on your Python path.
        \item In Python, use \lstinline[style=python]!import sys; print sys.path! to see what folders are on your path.
    \end{itemize}
    
    \medskip
    
    A cheatsheet (in the \texttt{doc} folder).
    \begin{itemize}
        \item Should get you started writing your own code.
        \item Compares plain CasADi vs. CasADi + \texttt{mpctools}.
    \end{itemize}
    
    \medskip
    
    A bunch of example files, e.g.,
    \begin{itemize}
        \item \texttt{nmpcexample.py}: Example of linear vs. nonlinear MPC.
        \item \texttt{cstr\_startup.py}: startup and a setpoint change (with no disturbances) for the CSTR system from Example 1.11.
    \end{itemize}
\end{frame}


%\begin{frame}[fragile]{Developers}
%
%A few extra steps if you want to be able to push changes:
%
%\begin{itemize}
%    \item Get Jim to authorize you
%    \item Go to \smallurl{https://my.cae.wisc.edu/}
%    \item Web Tools $\rightarrow$ Repositories
%    \item Set Alternate Repository Passwords
%    \item Add lines to \smallurl{.hg/hgrc}:
%\end{itemize}
%
%\begin{lstlisting}[style=shell]
%    [auth]
%    .prefix = hg.cae.wisc.edu/hg
%    .username = <username>
%    .password = <password>
%\end{lstlisting}
%
%\end{frame}


\begin{frame}{Why did we write this code?}

\begin{itemize}
    \item We plan to solve nonlinear MPC problems.
    \item CasADi is more robust than our \texttt{mpc-tools}
    \item However, setting up an MPC problem in CasADi takes a fair bit of code
    \item Everyone copy/pasting their own code is bad.
    \item A simpler interface means we (and others) can save a lot of time.
\end{itemize}

\end{frame}

\begin{frame}[fragile]{From official CasADi Examples}

\begin{columns}
    \begin{column}{.75\textwidth}
        
\begin{lstlisting}[style=python,basicstyle=\ttfamily\fontsize{6}{8}\selectfont]
# For all collocation points: eq 10.4 or 10.17 in Biegler's book
# Construct Lagrange polynomials to get the polynomial basis at
# the collocation point
for j in range(deg+1):
    L = 1
    for j2 in range(deg+1):
        if j2 != j:
            L *= (tau-tau_root[j2])/(tau_root[j]-tau_root[j2])
    
    lfcn = SXFunction([tau],[L])
    lfcn.init()
    # Evaluate the polynomial at the final time to get the
    # coefficients of the continuity equation
    lfcn.setInput(1.0)
    lfcn.evaluate()
    D[j] = lfcn.getOutput()
    
    # Evaluate the time derivative of the polynomial at all
    #collocation points to get the coefficients of the
    #continuity equation
    tfcn = lfcn.tangent()
    tfcn.init()
    for j2 in range(deg+1):
        tfcn.setInput(tau_root[j2])
        tfcn.evaluate()
        C[j][j2] = tfcn.getOutput()
\end{lstlisting}
    \end{column}
    \begin{column}{.25\textwidth}
        We don't want everyone writing this themselves!
    \end{column}
\end{columns}

\end{frame}

\begin{frame}[fragile]{Python Basics}
    For our purposes Python+Numpy isn't that much different from Octave/\textsc{Matlab}.
    
    \begin{center}
        \begin{tabular}{ll}
            \toprule
            Octave/\textsc{Matlab} & Python+Numpy \\
            \midrule
            \lstinline[style=matlab]!A = zeros(3,2);! & \lstinline[style=python]!A = np.zeros((3,2))! \\
            \lstinline[style=matlab]!x = ones(2,1); %  2D.! & \lstinline[style=python]!x = np.ones((2,)) #  1D.! \\
            \lstinline[style=matlab]!y = A*x;! & \lstinline[style=python]!y = A.dot(x)! \\
            \lstinline[style=matlab]!z = y.^2; %  Elementwise.! & \lstinline[style=python]!z = y**2 #  Elementwise.! \\
            \lstinline[style=matlab]!A(1,1) = 2; %  One-based.! & \lstinline[style=python]!A[0,0] = 2 #  Zero-based.! \\
            \lstinline[style=matlab]!A(2,:) = [3,4];! & \lstinline[style=python]!A[1,:] = np.array([3,4])! \\
            \lstinline[style=matlab]!s = struct('field',1);! & \lstinline[style=python]!s = {"field"  : 1}! \\
            \lstinline[style=matlab]!disp(s.field);! & \lstinline[style=python]!print s["field"]! \\
            \bottomrule
        \end{tabular}
    \end{center}
    
\end{frame}


\begin{frame}{General Tips}
    \begin{itemize}
        \item Remember that indexing is 0-based.
        \item Use NumPy's \texttt{array} instead of \texttt{matrix}.
        \begin{itemize}
            \item Despite what Octave/\textsc{Matlab} says, everything is \emph{not} a matrix of doubles
            \item Have to use \texttt{A.dot(x)} instead of \texttt{A*x}
            \item However, indexing is MUCH easier with arrays
        \end{itemize}
        \item Use \texttt{bmat([[A,B],[C,D]]).A} to assemble matrices from blocks.
        \begin{itemize}
            \item Equivalent to \texttt{[A, B; C, D]} in Octave/\textsc{Matlab}
            \item Trailing \texttt{.A} casts back to \texttt{array} type from \texttt{matrix}
        \end{itemize}
        \item Use \texttt{scipy.linalg.solve(A,b)} to compute $A^{-1}b$.
    \end{itemize}
\end{frame}

\begin{frame}[fragile]{System Model}

Start by defining the system model as a Python function.

\begin{lstlisting}[style=python]
def ode(x,u,d):
    # Grab the states, controls, and disturbance.
    [c, T, h] = x[:Nx]
    [Tc, F] = u[:Nu]
    [F0] = d[:Nd]
    
    # Now create the right-hand side function of the ODE.
    rate = k0*c*np.exp(-E/T)
    
    dxdt = [
        F0*(c0 - c)/(np.pi*r**2*h) - rate,
        F0*(T0 - T)/(np.pi*r**2*h)
        - dH/(rho*Cp)*rate
        + 2*U/(r*rho*Cp)*(Tc - T),
        (F0 - F)/(np.pi*r**2)
    ]
    
    return np.array(dxdt)
\end{lstlisting}
\end{frame}

\begin{frame}[fragile]{System Simulation}

The nonlinear system can be simulated using CasADi \texttt{integrator} objects with a convenient wrapper.
    
\begin{lstlisting}[style=python]
# Turn into casadi function and simulator.
ode_casadi = mpc.getCasadiFunc(ode,
    [Nx,Nu,Nd],["x","u","d"],funcname="ode")
cstr = mpc.DiscreteSimulator(ode, Delta, [Nx,Nu,Nd], ["x","u","d"])

# Simulate with nonlinear model.
x[n+1,:] = cstr.sim(x[n,:] + xs, u[n,:] + us, d[n,:] + ds) - xs
\end{lstlisting}

\end{frame}

\begin{frame}[fragile]{Calls to LQR and LQE}

\vspace{-10pt}
The functions \texttt{dlqr} and \texttt{dlqe} are also provided in \texttt{mpc-tools-casadi}.
\begin{columns}[T]

\begin{column}[T]{0.49\textwidth}
\begin{block}{Octave/\textsc{Matlab}}

\begin{lstlisting}[style=matlab]
% Get LQR.
[K, Pi] = dlqr(A, B, Q, R);

% Get Kalman filter.
[L, M, P] = dlqe(Aaug, ...
    eye(naug), Caug, Qw, Rv);
Lx = L(1:n,:);
Ld = L(n+1:end,:);

\end{lstlisting}

\end{block}
\end{column}

\begin{column}[T]{0.49\textwidth}
\begin{block}{CasADi/Python}

\begin{lstlisting}[style=python]
# Get LQR.
[K, Pi] = mpc.util.dlqr(A,B,Q,R)

# Get Kalman filter.
[L, P] = mpc.util.dlqe(Aaug,
    Caug, Qw, Rv)
Lx = L[:Nx,:]
Ld = L[Nx:,:]        


\end{lstlisting}

\end{block}
\end{column}

\end{columns}     

\end{frame}

\begin{frame}[fragile]{Controller Simulation}

\vspace{-1.5em}
\begin{columns}[T]

\begin{column}[T]{0.45\textwidth}
\begin{block}{Octave/\textsc{Matlab}}
\vspace*{-.75em}
\begin{lstlisting}[style=matlab,basicstyle=\ttfamily\fontsize{5pt}{6}\selectfont]
for i = 1:ntimes
  % Take plant measurement.
  y(:,i) = C*x(:,i) + v(:,i);
  
  % Update state estimate with measurement.
  ey = y(:,i) - C*xhatm(:,i) -Cd*dhatm(:,i);
  xhat(:,i) = xhatm(:,i) + Lx*ey;
  dhat(:,i) = dhatm(:,i) + Ld*ey; 
  
  % Steady-state target.
  H  = [1 0 0; 0 0 1];
  G = [eye(n)-A, -B; H*C, zeros(size(H,1), m)];
  qs = G\[Bd*dhat(:,i); ...
      H*(target.yset-Cd*dhat(:,i))];
  xss = qs(1:n); 
  uss = qs(n+1:end);
  
  % Regulator.
  u(:,i) = K*(xhat(:,i) - xss) + uss; 
  if (i == ntimes) break; end 
  
  % Simulate with nonlinear model.
  t = [time(i); mean(time(i:i+1)); time(i+1)];
  z0 = x(:,i) + zs;   F0 = p(:,i) + Fs;
  Tc = u(1,i) + Tcs;  F  = u(2,i) + Fs;
  [tout, z] = ode15s(@massenbal, t, z0, opts);
  x(:,i+1) = z(end,:)' - zs;
  
  % Advance state estimate.
  xhatm(:,i+1) = A*xhat(:,i) ...
      + Bd*dhat(:,i) + B*u(:,i);
  dhatm(:,i+1) = dhat(:,i);
end
\end{lstlisting}
\vspace{-.5em}
\end{block}
\end{column}

\begin{column}[T]{0.45\textwidth}
\begin{block}{Python+Numpy \vphantom{/}}
\vspace*{-.75em}
\begin{lstlisting}[style=python,basicstyle=\ttfamily\fontsize{5pt}{6}\selectfont]
for n in range(Nsim + 1):
    # Take plant measurement.
    y[n,:] = C.dot(x[n,:]) + v[n,:]
    
    # Update state estimate with measurement.
    err[n,:] = (y[n,:] - C.dot(xhatm[n,:])
        - Cd.dot(dhatm[n,:]))
    xhat[n,:] = xhatm[n,:] + Lx.dot(err[n,:])
    dhat[n,:] = dhatm[n,:] + Ld.dot(err[n,:])
    
    # Make sure we aren't at the last timestep.
    if n == Nsim: break

    # Steady-state target.
    rhs = np.concatenate((Bd.dot(dhat[n,:]),
        H.dot(ysp[n,:] - Cd.dot(dhat[n,:]))))
    qsp = linalg.solve(G,rhs) # i.e. G\rhs.
    xsp = qsp[:Nx]
    usp = qsp[Nx:]
    
    # Regulator.
    u[n,:] = K.dot(xhat[n,:] - xsp) + usp
    
    # Simulate with nonlinear model.
    x[n+1,:] = cstr.sim(x[n,:] + xs,
        u[n,:] + us, d[n,:] + ds) - xs
    
    # Advance state estimate.
    xhatm[n+1,:] = (A.dot(xhat[n,:])
        + Bd.dot(dhat[n,:]) + B.dot(u[n,:]))
    dhatm[n+1,:] = dhat[n,:]
                       
\end{lstlisting}
\vspace{-.5em}
\end{block}
\end{column}

\end{columns}     

\end{frame}

\begin{frame}{Results}

\begin{columns}
    \begin{column}{.475\textwidth}
        \begin{block}{Octave/\textsc{Matlab}}
            \begin{center}
                \includegraphics[width=\textwidth]{cstr_octave.pdf}
            \end{center}
        \end{block}
    \end{column}
    
    \begin{column}{.475\textwidth}
        \begin{block}{Python+Numpy \vphantom{/}}
            \begin{center}
                \includegraphics[width=\textwidth]{cstr_python.pdf}
            \end{center}
        \end{block}
    \end{column}
\end{columns}

\end{frame}


\begin{frame}{What can we do with \texttt{mpc-tools-casadi}?}
    \begin{itemize}
        \item Discrete-time linear MPC
        \item Discrete-time nonlinear MPC
        \begin{itemize}
            \item Explicit models
            \item Runge-Kutta discretization
            \item Collocation
        \end{itemize}
        \item Discrete-time nonlinear MHE
        \begin{itemize}
            \item Explicit models
            \item Runge-Kutta discretization
            \item Collocation
        \end{itemize}
        \item Basic plotting function
        \item Example scripts
        \begin{itemize}
            \item Linear
            \item Solution of linear as nonlinear
            \item Periodic linear
            \item Example 2-8
            \item Simple collocation
            \item Example 1-11
        \end{itemize}
    \end{itemize}
    
\end{frame}

\begin{frame}{Example: Van der Pol Oscillator}
For this problem, nonlinear MPC performs slightly better.
\begin{itemize}
    \item The computation isn't much more time-consuming because of the power of Casadi.
    \item The problem isn't difficult to set up because of \texttt{mpc-tools-casadi}.
\end{itemize}
    
\begin{columns}
    \begin{column}{.475\textwidth}
        \begin{block}{Linear MPC}
            \begin{center}
                \includegraphics[width=\textwidth]{vdposcillator_lmpc.pdf}
            \end{center}
        \end{block}
    \end{column}
    \begin{column}{.475\textwidth}
        \begin{block}{Nonlinear MPC}
            \begin{center}
                \includegraphics[width=\textwidth]{vdposcillator_nmpc.pdf}
            \end{center}
        \end{block}
    \end{column}
\end{columns}
\end{frame}

\begin{frame}{More Complicated Example}

Using \texttt{mpc-tools-casadi}, we can replace the LQR and KF from Example 1.11 with nonlinear MPC and MHE.

\begin{itemize}
    \item \texttt{cstr\_startup.py} shows basic structure and a setpoint change.
    \item \texttt{cstr\_nmpc\_nmhe.py} shows steady-state target finding and NMHE.
    \item See the cheatsheet for important functions and syntax.
\end{itemize}

\end{frame}

\begin{frame}{\texttt{cstr\_startup.py}}
    Here, nonlinear MPC knows to be less aggressive.
    \begin{center}
        \includegraphics[width=.75\textwidth]{cstr_startup.pdf}
     \end{center}
\end{frame}


\begin{frame}{What can't we do yet?}
    \begin{itemize}
        \item True continuous-time formulation
        \begin{itemize}
            \item Continuous-time models with explicit time dependence are not supported
            \item Quadrature for continuous-time objective function is available via collocation
            \item DAE systems are possible in principle
        \end{itemize}
        \item Quality guess generation
        \begin{itemize}
            \item Solve sequence of smaller problems
            \item Use as initial guess for large problem
            \item Must do ``by hand''
        \end{itemize}
    \end{itemize}
\end{frame}

%\begin{frame}{Additional Notes}
%
%\begin{itemize}
%    \item We may not be able to write good CasADi code, but we at least want to write good Python code
%    \begin{itemize}
%        \item \smallurl{http://docs.python-guide.org/en/latest/writing/style/}
%        \item \smallurl{https://www.python.org/dev/peps/pep-0008/}
%    \end{itemize}
%    \item NumPy is verbose and takes getting used to, especially if you're using matrices
%    \item Duck-typing makes me nervous, but it's kind of a major paradigm of Python
%    \begin{itemize}
%        \item Lists vs. NumPy vectors
%        \item NumPy matrices vs. arrays
%        \item CasADi \texttt{MX} vs. \texttt{SX}
%    \end{itemize}
%    \item ``Easier to Ask Forgiveness than Permission'' is challenging if you don't know what exceptions might be raised
%    \begin{itemize}
%        \item Takes some getting used to if you've been raised on ``Look Before You Leap'' (common in Octave/\textsc{Matlab})
%        \item A brief discussion: \smallurl{http://code.activestate.com/lists/python-list/337643/}
%    \end{itemize}
%\end{itemize}
%
%\end{frame}

\end{document}
